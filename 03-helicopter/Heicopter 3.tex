\documentclass{article}
\usepackage{amsmath}
\usepackage{graphicx}
\usepackage{hyperref}

\title{Simulation of Helicopter Proportional Control System with Controller}
\author{Yousef Al-Hadhrami}
\date{18.02.2025}

\begin{document}

\maketitle

\section{Objective}
The objective of this simulation is to design and analyze a proportional feedback control system for a helicopter's yaw control using Ptolemy II. The system aims to regulate the helicopter’s yaw rate by applying control torque proportional to the discrepancy between desired and actual yaw rates.

\section{Mathematical Model}
The helicopter's yaw dynamics are modeled as a second-order rotational system with the following parameters:
\begin{itemize}
    \item \textbf{Moment of Inertia:} $I_{yy} = 3800 \ kg \cdot m^2$
\end{itemize}

The governing equations are:
\begin{align}
    e(t) &= \psi(t) - \dot{\theta}(t) \\
    T_y(t) &= K \cdot e(t) \\
    \dot{\theta}(t) &= \dot{\theta}(0) + \frac{1}{I_{yy}} \int_0^t T_y(\tau) d\tau
\end{align}

\section{Model Setup in Ptolemy II}
\subsection{Main Graph Configuration}
\begin{figure}
    \centering
    \includegraphics[width=0.5\linewidth]{mainGraph.png}
    \caption{Main Graph}
    \label{fig:enter-label}
\end{figure}
\begin{itemize}
    \item \textbf{Continuous Director:} Runs for 100 seconds.
    \item \textbf{Actors Used:}
    \begin{itemize}
        \item \textbf{CurrentTime:} Provides the current simulation time.
        \item \textbf{MonitorTime:} Displays the simulation time.
        \item \textbf{Helicopter Actor:} Processes the control input and outputs yaw angle (Theta) and yaw rate (ThetaDot).
        \item \textbf{Controller Actor:} Computes the required control signal.
        \item \textbf{Constant:} Provides the command signal $\psi = 0.2$.
        \item \textbf{Monitor Actors:} Used to display values of Theta, ThetaDot, and Command.
    \end{itemize}
\end{itemize}

\subsection{Connections in the Main Graph}
\begin{itemize}
    \item $\dot{\theta}$ from Helicopter $\rightarrow$ $\dot{\theta}_{receive}$ in Controller.
    \item Constant Command (0.2) $\rightarrow$ $CommandValueReceive$ in Controller.
    \item $CommandSend$ from Controller $\rightarrow$ Receiver in Helicopter.
    \item $CommandSend$ from Controller $\rightarrow$ MonitorCommand.
\end{itemize}

\section{Controller Composite Actor Setup}
\begin{figure}
    \centering
    \includegraphics[width=0.5\linewidth]{Controller Composite Actor.png}
    \caption{Controller Composite Actor}
    \label{fig:enter-label}
\end{figure}
The controller calculates the error between the command and actual yaw rate, then applies proportional control:
\begin{itemize}
    \item \textbf{AddSubtract Block:}
    \begin{itemize}
        \item Input 1 (Add): $CommandValueReceive$ 
        \item Input 2 (Substract): $ThetaDotReceive$ 
        \item Output: Error $e(t)$
    \end{itemize}
    \item \textbf{MultiplyDivide Block:}
    \begin{itemize}
        \item Input 1 (Multiply): Error $e(t)$
        \item Input 2 (Multiply): Constant 850 (Gain $K$)
        \item Output: Control Signal $T_y(t)$
    \end{itemize}
    \item \textbf{Control Signal Output:} Sent to the helicopter’s receiver.
\end{itemize}

\section{Helicopter Composite Actor Setup}
\begin{figure}
    \centering
    \includegraphics[width=0.5\linewidth]{Helicopter Composite Actor.png}
    \caption{Helicopter Composite Actor}
    \label{fig:enter-label}
\end{figure}
The helicopter processes the control signal to determine yaw dynamics:
\begin{itemize}
    \item \textbf{Integrator Block:} Converts control torque into angular acceleration.
    \item \textbf{MultiplyDivide Block:}
    \begin{itemize}
        \item Input 1 (Multiply): Integrated torque (from Integrator)
        \item Input 2 (Divide): Constant 3800 (Moment of Inertia $I_{yy}$)
        \item Output: Angular velocity $\dot{\theta}$
    \end{itemize}
    \item \textbf{Second Integrator Block:} Converts $\dot{\theta}$ to yaw angle $\theta$.
    \item \textbf{Outputs:} $\theta$ and $\dot{\theta}$ displayed using monitor blocks.
\end{itemize}

\section{Simulation Results}
The simulation results for 100 seconds show:
\begin{itemize}
    \item \textbf{Final Yaw Angle (Theta):} 19.105 rad
    \item \textbf{Final Yaw Rate (ThetaDot):} 0.199999 rad/s
    \item \textbf{Final Control Command:} 3.25289E-8
\end{itemize}

\subsection{Performance Analysis}
\begin{itemize}
    \item The system achieves a steady-state yaw rate of \textbf{0.2 rad/s}, matching the reference input.
    \item The yaw angle follows a linear increase, as expected in a proportional control system.
    \item The control system exhibits minimal error at steady-state.
\end{itemize}

\section{Conclusion}
This simulation successfully demonstrates a proportional feedback control system for a helicopter's yaw dynamics using Ptolemy II. The results validate the effectiveness of the control model in achieving the desired yaw rate while maintaining stability. Future work may involve implementing a \textbf{PID controller} to improve transient response and reduce steady-state error.

\end{document}